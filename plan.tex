\documentclass[a4paper,10pt,english]{article}
\usepackage[T1]{fontenc}
\usepackage{geometry}
\geometry{verbose,tmargin=3cm,bmargin=3cm,lmargin=3cm,rmargin=3cm}
\usepackage[backend=bibtex]{biblatex}
\addbibresource{sources.bib}
\usepackage{babel}


\title{Research Plan for MAPF with matching using ICTS (draft)}
\author{Thom van der Woude}
\date{\today}

\begin{document}
\maketitle


\section*{Background of the research}
The Dutch Railways (NS) company needs to clean and service their fleet of trains during the night in shunting yards so that they can properly bring passengers from A to B during the day. The problem of scheduling the trains to achieve this is the Train Unit Shunting and Servicing (TUSS) problem \cite{mulderij2020} and it is a typical NP-hard problem. To gain more insight into the characteristics of this many-sided problem and in particular the feasibility of its instances, a relaxation is considered: the multi-agent path finding problem extended with a matching sub-problem or matching multi-agent pathfinding (hereafter MMAPF). 
\paragraph{Multi-agent pathfinding} MAPF is a well-studied problem, a general overview is given in \cite{stern2019}. Somewhat informally, it is the problem of routing multiple agents on a graph (often a 4-connected grid) to multiple distinct destinations without any conflicts, with definitions of what constitutes a conflict varying; in this project, conflict is taken to mean 'occupy the same node' or 'traverse the same edge in opposite directions at the same time-step'. Typically one of two things is minimized: the sum of individual costs (SIC), or the length of the longest path (also known as \textit{makespan}). Algorithms for MAPF are often designed to find optimal solutions so that for instance the SIC is minimal.

\paragraph{Matching Multi-agent pathfinding as TUSS relaxation}
For a problem to be a relaxation of TUSS, it has to be at most as difficult as TUSS for each instance assuming some suitable mapping from the original problem to the relaxation. In one respect TUSS is somewhat easier to solve than classical MAPF: trains are partitioned into types (equivalence classes, often denoted by colours) and destinations are labelled as destinations for trains of one specific type. This means that as far as this matching is concerned, for any valid matching there can be many equivalent permutations, so that in different optimal solutions a single train might have different destinations. As a consequence, this matching has to be modelled in the relaxed problem, resulting in MMAPF. Contrast this variation with both normal MAPF where each agent only has one target and anonymous MAPF where agents can be assigned to any target. See \cite{ma2016} for a discussion of MAPF with matching.

\paragraph{Solving MAPF with matching}
There exist many different algorithms for solving the MAPF problem. 
Some current examples include conflict-based search\cite{sharon2015}, branch-and-cut-and-price\cite{lam2019}, A* with Independence detection and Operator Decomposition\cite{standley2010}, M*\cite{wagner2011} and Increasing Cost Tree Search\cite{sharon2011}. 
None of these are designed to work with matching out-of-the-box, so a naive way to incorperate matching into MAPF would be to generate legal matchings in preprocessing and enumerate these until one is found for which MAPF finds a solution. 
However, the challenge of this research project is to find ways using heuristics, preprocessing or other methods to generate good matchings that result in feasible MAPF problems and ideally efficient MAPF solutions. 
Similar to a previous iteration of this research project, we each chose an algorithm to work with as a start of our respective subprojects: I will be working on incorperating matching into Increasing Cost Tree Search MAPF-solving. 

\paragraph{TUSS to MAPF with matching}
Finding a suitable mapping between TUSS and MMAPF instances is an unsolved problem: intuitively it is desirable to preserve feasibility so that there can never be a feasible TUSS instance mapped to an infeasible MMAPF instance, but perhaps it should be acceptable for the mapping to turn infeasible TUSS instances into feasible MMAPF instances, as the latter problem is a relaxation of the former. In any case, a good mapping that preserves key instance properties is important for MMAPF to be practically useful as a relaxation of TUSS so should time allow this would be interesting to consider.


\section*{Research Question}
The key research question that I aim to answer in this subproject is this: How can the Increasing Cost Tree Search method be used or adapted to efficiently find solutions to Multi-agent path-finding with matching? Relevant subquestions are as follows, where matching is to be understood as a single legal matching of trains to targets (departures) of their respective types:\footnote{These subquestions were taken from the initial project proposal at \url{https://projectforum.tudelft.nl/course_editions/39/projects/977}}
\begin{enumerate}
	\item Can we use a heuristic or preprocessing step to calculate a matching? And can we do so such that the resulting instance with the fixed matching is feasible?
	\item Can we use instance features to (help) determine an efficient matching?
	\item How can we improve upon a matching once one has been found?
\end{enumerate}

\section*{Method}
As a first step, an existing implementation of Increasing Cost Tree Search has to be found or a new one has to be implemented following the paper if no algorithm is publicly available. Next, the generation of agent-target matchings is to be implemented, which are translated into MAPF instances to be solved using ICTS. It might be possible to collaborate within the peer group here as we all have to do this step. Once this task has been completed, the project takes a more theoretical turn and I intend to study literature specific to MAPF with colours/teams or similar problems with the additional matching constraint in order to work towards answering the subquestions. Here also I hope to collaborate with my peer group and exchange ideas, perhaps about how properties of instances might be exploited in finding matchings or similar concepts.

\section*{Planning of the (first week of the) research project} 
For now I assume that reading a paper, looking into relevant citations and making a summary for comprehension takes around 3-4 hours which is also reflected in my planning. Also note that I take the tasks this first week to be seperate from those to be described by the end of this week in Method.
\begin{enumerate}
	\item[19-4 ] 
	\begin{enumerate}
		\item[10:45] Start of research project
		\item[12:45] Meeting with peer group
		\item[13:45] Assignment of subproject (ICTS)
		\item[15:45] Planning of first week (deadline at 19-4 23:59).
	\end{enumerate}
	\item[20-4] 
	\begin{enumerate}
	\item[9:00] Revise information literacy, find relevant conferences/journals for this research project
	\item[10:00] Read and summarize \cite{stern2019}
	\item[15:00] Read and summarize \cite{mulderij2020}
	\end{enumerate}
	\item[21-4]
		\begin{enumerate}
		\item[9:00] Read and summarize \cite{sharon2011}, the original ICTS paper.
		\item[13:00] Read and summarize \cite{standley2010}
		\item[15:00] Look into other important citations in \cite{sharon2011}
	\end{enumerate}
	\item[22-4] \begin{enumerate}
		\item[9:00] Look into possible existing ICTS implementations that can be used (important for planning).
		\item[10:00] Survey MAPF literature in a breadth-first way in order to flesh out background of research. Also think about research question formulation and corresponding methods.
		\item[13:00] Meet with group and supervisor on Teams.
		\item[14:00] Continue planning the research.
	\end{enumerate}
	\item[23-4]\begin{enumerate}
		\item[9:00]  Finish research plan (official deadline is at 25-4 23:59 but target deadline at 23-4 23:59). If there is time remaining, start familiarizing with the MAPF web API that our peer group 'inherited' from the previous instance of this project, and start implementing ICTS if no suitable existing implementation was found the previous day. (note: this is just one block as I am not sure how long this will take given that planning is rather hard)
	\end{enumerate}
\end{enumerate}
\printbibliography

\end{document}
