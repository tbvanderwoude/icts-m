\documentclass[english]{article}
\usepackage{geometry}
\usepackage{hyperref}
\usepackage{inputenc}
\geometry{verbose,tmargin=3.5cm,bmargin=4cm,lmargin=3.8cm,rmargin=3.8cm}
\usepackage[backend=biber,style=ieee]{biblatex}
\addbibresource{sources.bib}
\makeatletter
\usepackage{url}
\usepackage{graphicx}
\usepackage{caption}
\usepackage{wrapfig}
\usepackage{subcaption}
\makeatother
\usepackage{babel}

\begin{document}
	
	\title{Dump}
	
	\author{Thom van der Woude}
	\date{}
	
	\maketitle

	\section{Introduction}
					
			%Comparatively little research has been done on MAPFM and solving it optimally. 
			%For MAPFM with a makespan objective instead of SIC, known as the combined target-assignment and pathfinding (TAPF) problem, there is the Conflict-Based Min-Cost-Flow (CBM) algorithm by Ma and Koenig \cite{ma2016} and a similar approach described in \cite{henkel2019}.
			%Ma and Koenig's method uses the Meta-Agent variant of CBS \cite{sharon2015} as a high-level search framework; in the low-level search, single teams are planned in polynomial time as constrained anonymous MAPF instances using a max-flow approach \cite{yu2013}. 
			
			%The TAPF problem \cite{ma2016} combines the target-assignment and the pathfinding problems into one optimization problem, the objective being minimal makespan.
			
			%In order to find the
			
			%but this comes at the price of optimality
			
			%have teams of agents with a pool of goals assigned to them. 
			
			%targets are often given to teams of agents in real-world applications such as warehouse robotics 
			
			%In a MAPF instance, each agent is assigned a single goal to navigate to. However, w
			
			
			
			
			
			
			
			%This means that to prevent conflicts, agents have to be planned 
			%
			% g underpins many real-world
			%The single-agent pathfinding
			%Multi-agent pathfinding (MAPF) is the problem 
			%The Dutch Railways company (NS) needs to clean and service their fleet of trains during the night in shunting yards so that they can properly bring passengers from A to B during the day. 
			%The problem of scheduling the trains and personnel to achieve this is called the Train Unit Shunting and Servicing (TUSS) problem \cite{mulderij2020}. 
			%It is an NP-hard problem with many different subproblems in addition to the basic train unit routing, such as the timetabling of personnel and coupling and decoupling of train units entering and leaving the yard. 
			% introduce heuristics here.
			
			%In practical scenarios, only heuristic methods can solve TUSS instances \cite{geiger2018}. Such a heuristic is used at NS to determine the capacity of shunting yards, which is the number of units a yard can serve in a day. 
			%This is done by defining a set of instances for the yard and determining the number of units for which 95\% of instances can be solved by the heuristic within a set time-out \cite{ven2019}. 
			%A problem is that a subset of feasible TUSS instances is not solved by the heuristic, meaning that the found capacity can be lower than the true capacity. 
			%Combined with the fact that current upper-bounds for capacity are highly optimistic, this makes it difficult to decide when to expand shunting infrastructure. 
			%Mulderij et al. propose the multi-agent pathfinding (MAPF) problem, as described in \cite{stern2019}, extended with a matching subproblem (hereafter MAPFM) as a suitable relaxation of (discretised) TUSS for finding better upper-bounds for capacity using an approach analogous to that outlined above \cite{mulderij2020}.
			%For this to succeed, a way to optimally solve MAPFM instances is needed, where optimal is defined as minimizing the Sum of Individual Costs (SIC) objective.
		
		
		each give (loose) upper bounds on the optimal solution cost for each instance.
		
		so if a solution is found its cost is an upper bound for the optimal solution cost; however, if no solution can be found, 
		
		; the optimal solution cost cannot generally be known, 
		While solutions to instances found using heuristics are assumed to be suboptimal, it is not possible to estimate by how much 
		
		As a result, found solutions are assumed to be suboptimal, they currently cannot 
		
		Still, such methods will not find a solution to every instance in reasonable time,
		Heuristic solutions to instances defined for a set shunting yard can be said to characterise a lower bound on the capacity of the shunting yard.  
		
		This is because heuristic methods are suboptimal and not guaranteed to be complete, meaning that no solution may be found for a feasible instance and if one is found its solution cost might far exceed the optimal cost.  
		
		
		---
			this leaves railway companies in the dark about 
		
		when shunting infrastructure has to be exp whether shunting infrastructure has to be expanded to accomodate for increasing     
		
		As a result, because current upper-bounds for capacity are highly optimistic, 
		
		
		meaning that it is hard for railway companies to 
		
		As optimal solution costs cannot be known \cite{geiger2018}, 
		The railway company, however, is interested in learning tight upper bounds for the capacity of existing shunting yards to inform decisions about matters like infrastructure expansion. 
	\paragraph{Old}
	MAPFM is the problem of finding agent-goal matchings that induce a non-conflicting routing of agents to their goals. Each matching corresponds to a MAPF instance, so each feasible matching has an optimal solution, i.e. a solution with a minimal Sum of Individual Costs (SIC). In approaching MAPFM as a TUSS relaxation to find upper-bounds for instances, optimal MAPFM solutions are the most interesting as they can characterise such upper-bounds. Furthermore, defining optimality as minimal SIC corresponds more closely to TUSS objective functions such as personnel cost than the other common MAPF objective, minimal makespan. Optimally solving MAPFM is equivalent to finding a feasible matching such that there is no other matching for which the corresponding MAPF instance has a lower optimal SIC. This motivates the joint nature of MAPFM: some matchings might be infeasible and others may have sub-optimal SIC costs, so especially in larger instances where there may be thousands of matchings, it is preferable to integrate the generation of matchings with the search for non-conflicting path-combinations to avoid having to exhaustively process all matchings. In this work, both integrated approaches and exhaustive methods using ICTS are considered.
		
	\subsection{Formulation}
	Formally, a MAPFM instance can be described as follows. Let $G = (V,E)$ be an undirected connected graph, with each $v\in V$ representing an obstacle-free tile on a 4-grid and each $e = (u,v)\in E$ representing a legal uniform-cost move between two such tiles. Let there be $K$ teams $t_1,\ldots, t_K$, where $t_i$ consists of $k_i$ agents with start positions $s_1^i,\ldots,s_{k_i}^i$ together with an equal number of team goals $g_1^i,\ldots,g_{k_i}^i$. 
	
	Within each team $t_i$, agent $a_j^i$ are to be \textit{matched} to a unique goal $g_k^i$, so that exactly one agent is assigned to each goal within team $i$; once all agents are matched to goals, for each agent $a_j^i$ matched to $g_k^i$, a path from $s_j^i$ to $g_k^i$ is to be found such that all agents paths taken together are non-conflicting and therefore make up a solution \cite{ma2016}. In this work, non-conflicting means, in the conflict-terminology of \cite{stern2019}, that there are no vertex conflicts (and thus no edge conflicts) and no swapping conflicts. Practically, this means that no two agents may be at $v\in V$ at the same timestep and that if agents $i,j$ are at locations $u,v$ respectively at timestep $t$, they may not be at locations $v,u$ at timestep $t+1$. 
	\printbibliography
	
\end{document}
