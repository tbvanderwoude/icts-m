%Universal prelude for everything sciency, mathy, and what have you
%Make semantic macros, not syntactic macros!
%maths stuff
\usepackage[stretch=10]{microtype}
\usepackage{todonotes}
\usepackage{amsfonts,amsmath, amssymb,amsthm,steinmetz}
\usepackage[utf8]{inputenc}
%nice tables
\usepackage{booktabs}
\usepackage{multicol}
\usepackage{chemfig}
\usepackage{lmodern}
\usepackage{enumitem}
%units for doing a science. allowlitunits means things like 20\milli\meter in math mode without \SI{}{}
\usepackage[allowlitunits]{siunitx}
%time, you always need time
\usepackage{datetime}
%semi-nice karnaugh maps which can be defined using minterms, maxterms etc.
\usepackage{karnaugh-map}
\newtheorem{theorem}{Theorem}
\usepackage{marginnote}
\usepackage{tikz}
\usepackage{subcaption}
\usepackage[normalem]{ulem}
\usepackage{graphicx} 
\usetikzlibrary{quotes,angles}
%\dv <3
\usepackage{physics}
%bra-ket notation for quantum stuff
\usepackage{braket}
%for surface and volume integrals
\usepackage{esint}
\usepackage{enumitem}
%Specific hackery/macros to save time/make latex more semantic less syntactic
\DeclareMathOperator{\sinc}{sinc}
\newcommand{\conj}[1]{\,\overline{\!{#1}}}
\newcommand{\phasor}[1]{\,\widetilde{\!{#1}}}
\newcommand{\risingfac}[1]{%
	^{\overline{#1}}%
}
%for sets generating groups
\newcommand{\gen}[1]{%
	\langle #1\rangle
}
\newcommand{\fallingfac}[1]{%
	^{\underline{#1}}%
}
%Theorem style for "[Name of famous mathematician]'s Theorem" type of theorems, of which there are a surprising number.
\theoremstyle{named}
\newtheoremstyle{named}{}{}{\itshape}{}{\bfseries}{.}{.5em}{\thmnote{#3's }#1}
\newcommand{\bitvector}[2]{
	\underline{#1} = (#1_{#2-1},#1_{#2-2},...,#1_0)
}
\newcommand{\code}[1]{
	\texttt{#1}
}
\usepackage[leqno]{mathtools}
\usepackage{chngcntr}
\counterwithin{equation}{section}
\usepackage[roman, thin, thinp, thinc]{esdiff}

\usepackage{bm}
\usepackage{marginnote}
\usepackage{epigraph}
\usepackage{hyperref}