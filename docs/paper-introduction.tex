\documentclass[a4paper,10pt,english]{article}
\usepackage[T1]{fontenc}
\usepackage{geometry}
\geometry{verbose,tmargin=3cm,bmargin=3cm,lmargin=3cm,rmargin=3cm}
\usepackage[backend=biber,style=ieee]{biblatex}
\addbibresource{sources.bib}
\usepackage{babel}

\title{Multi-Agent Path Finding with matching using Increasing Cost Tree Search}
\author{Thom van der Woude}
\newcommand{\namelistlabel}[1]{\mbox{#1}\hfil}
\newenvironment{namelist}[1]{%1
	\begin{list}{}
		{
			\let\makelabel\namelistlabel
			\settowidth{\labelwidth}{#1}
			\setlength{\leftmargin}{1.1\labelwidth}
		}
	}{
\end{list}}
\date{\today}

\begin{document}
	\maketitle
	\begin{namelist}{xxxxxxxxxxxxxxxxxxxxxxxxxxxxxxxxxxxxxxx}
		\item[{\bf Title:}]
		Multi-Agent Path Finding with matching using Increasing Cost Tree Search
		\item[{\bf Author:}]
		Thom van der Woude
		\item[{\bf Responsible Professor:}]
		Mathijs de Weerdt
		\item[{\bf Other Supervisor:}]
		Jesse Mulderij
		\item[{\bf Peer group members:}]
		Robbin Baauw, Jonathan Dönszelmann, Ivar de Bruin, Jaap de Jong
	\end{namelist}
	
	\section*{Introduction}
	The Dutch Railways (NS) company\footnote{\url{https://www.ns.nl/}} needs to clean and service their fleet of trains during the night in shunting yards so that they can properly bring passengers from A to B during the day. The problem of scheduling the trains to achieve this is called the Train Unit Shunting and Servicing (TUSS) problem \cite{mulderij2020}. It is an NP-hard problem with many different subproblems in addition to the basic routing, such as the scheduling of personnel and coupling and decoupling of train units entering and leaving the yard. The company is interested in learning tighter upper-bounds for the capacity of their shunting yards but a problem is that in practical scenarios, only heuristics can solve TUSS instances (as was shown by Geiger et al. \cite{geiger2018}), which gives little information about what optimal solutions would look like and what could still be achieved using existing shunting infrastructure. To gain more insight into the capacity of shunting yards and the feasibility of problem instances, relaxations of TUSS can be useful. In \cite{mulderij2020}, the the multi-agent pathfinding (MAPF) problem extended \cite{stern2019} with a matching subproblem (hereafter MAPFM) is given as such a relaxation. Comparatively little research has been done on MAPFM, and the Conflict-Based Min-Cost-Flow algorithm by Ma and Koenig \cite{ma2016} is one of the few existing approaches to solving this more general problem along with the somewhat similar one described in \cite{henkel2019}. Ma and Koenig's method builds upon the conflict-based search method and in particular on the Meta-Agent variation thereof also discussed in \cite{sharon2015} as a high-level search framework, while exploiting the connection between MAPF problems and max-flow problems (as discussed in \cite{yu2013}) in the low-level search. This work introduces a novel algorithm for solving MAPFM based on the Increasing Cost Tree Search (ICTS) algorithm for MAPF \cite{sharon2011}. ICTS, in short, is a two-level approach to MAPF with a top-level breadth-first traversal of an Increasing Cost Tree (ICT) representing combinations of per-agent path-lengths, and a bottom-level evaluation of ICT nodes using multi-value decision diagrams (MDDS) that represent for each agents all possible paths to the goal of a set length.
	
	Pathfinding is a problem with many applications ranging from navigation and network routing to solving puzzles like Sokoban \cite{junghanns1999}. When paths for different agents can conflict, they need to be planned together to find a non-conflicting combination, giving rise to the multi-agent pathfinding problem \cite{stern2019}. There exist two main classes of agent-goal assignments for MAPF in the literature: anonymous MAPF \cite{yu2013} wherein agents can be matched to any goal and non-anonymous or regular MAPF where each agent has a single matching goal. A further generalization of this is the target assignment and path finding (TAPF) problem described in \cite{ma2016}, wherein both agents and goals are partitioned into same-sized teams and agents can be matched to any goal of their team. TAPF is a relaxation of the Train Unit Shunting and Servicing (TUSS) problem \cite{mulderij2020}, the problem of scheduling the cleaning and servicing of trains during the night in shunting yards. Specifically, in TUSS, each train unit has a type and a goal, 
	
	\newpage
	\printbibliography
\end{document}
