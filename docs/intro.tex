\documentclass[11pt]{report}
\title{Notes for intro}
\author{Thom van der Woude}
\date{\today}
\begin{document}
 \maketitle
 \section{Introduction}
 This course is something new. What is research? A research cycle: read-write-rinse-repeat.
 \paragraph{On research}
 'The systematic investigation into and study of materials and sources in order to establish facts and reach new conclusions.'

 You investigate a question, you establish a fact (maybe reproduce facts), reach conclusions.

 Failure is part of research.

 Ask good questions, many of them. Find answers to these. Communicate the answers you found.

 You don't have to do the module if you have already done it. But it is good to do anyway.
 
 \paragraph{Research skills}
 You need to know how to find information but also how to review/think about literature and work of peers. Also responsibly developing/investigating and communicating your entire proces and your results.


 Deliverables are always individual.



 Wat is het? Wat zijn bestaande variaties? Implementatie bouwen. Dan kijken naar matching.
 
\section{Information literacy}

\paragraph{Analysing}
Efficient searching for scientific information starts with analysing. You analyse your assignment and the background information that you have in order to formulate a general research topic. You split your research topic into sub-questions, which you then use to create a search plan.
 
With a good search plan, you know what you are looking for and you can be reasonably sure that you won't miss any important information. It ensures that your search results are relevant and complete.


Mind maps for overview?


\paragraph{Searching and evaluating}
After formulating your search query, you still need to determine what kind of information you're actually looking for, which information source you should use and how to search in this information source.
	
Evaluate your search by looking at the number and relevance of the results and you check the scientific nature of the results. Modify your search query if necessary.
\paragraph{Processing}
In order to use the publications you found, you first need to save them. Think about what your goal is when you save them:

to read them yourself
to share them with others
to generate a literature list
or a combination of the above

\end{document}
