\documentclass[english]{article}
\usepackage{geometry}
\usepackage{inputenc}
\geometry{verbose,tmargin=3.5cm,bmargin=4cm,lmargin=3.8cm,rmargin=3.8cm}
\usepackage[backend=biber,style=ieee]{biblatex}
\addbibresource{sources.bib}
\makeatletter
\usepackage{url}
\makeatother
\usepackage{babel}

\begin{document}

\title{Multi-Agent Path Finding with matching using Increasing Cost Tree Search}

\author{Thom van der Woude}
\date{\today}

\maketitle


\section{Introduction}The Dutch Railways (NS) company\footnote{\url{https://www.ns.nl/}} needs to clean and service their fleet of trains during the night in shunting yards so that they can properly bring passengers from A to B during the day. The problem of scheduling the trains to achieve this is called the Train Unit Shunting and Servicing (TUSS) problem \cite{mulderij2020}. It is an NP-hard problem with many different subproblems in addition to the basic routing, such as the scheduling of personnel and coupling and decoupling of train units entering and leaving the yard. The company is interested in learning tighter upper-bounds for the capacity of their shunting yards but a problem is that in practical scenarios, only heuristics can solve TUSS instances (as was shown by Geiger et al. \cite{geiger2018}), which gives little information about what optimal solutions would look like and what could still be achieved using existing shunting infrastructure. To gain more insight into the capacity of shunting yards and the feasibility of problem instances, Mulderij et al. \cite{mulderij2020} propose the multi-agent pathfinding (MAPF) problem \cite{stern2019} extended with a matching subproblem (hereafter MAPFM) as a suitable relaxation that can be used to compute upper-bounds for TUSS instances by solving the corresponding MAPFM instances optimally.

Comparatively, little research has been done on MAPFM, and the Conflict-Based Min-Cost-Flow algorithm by Ma and Koenig \cite{ma2016} is one of the few existing approaches to optimally solve this more general problem along with the somewhat similar one described in \cite{henkel2019}. Ma and Koenig's method builds upon the conflict-based search method and in particular on the Meta-Agent variation thereof also discussed in \cite{sharon2015} as a high-level search framework, while exploiting the connection between MAPF problems and max-flow problems (as discussed in \cite{yu2013}) in the low-level search. This begs the question: can alternative MAPF algorithms such as Increasing Cost Tree Search (ICTS) \cite{sharon2011} or M* \cite{wagner2011} also serve as the basis for a MAPFM solver?

This work is the outcome of a search for an efficient algorithm for optimally solving MAPFM derived from ICTS as an alternative to the approach outlined above\footnote{To the knowledge of the author, to date, no such algorithm is described in the literature}. ICTS, in short, is a two-level approach to MAPF with a top-level breadth-first traversal of an Increasing Cost Tree (ICT) representing combinations of per-agent path-lengths, and a bottom-level evaluation of ICT nodes using multi-value decision diagrams (MDDS) that represent per agent all possible paths to the goal of a set length. In the context of a research project in which multiple such MAPF algorithms were taken as starting points for MAPFM algorithms within a peer group, the novel ICTS-based algorithm is compared to these algorithms on a set of standard benchmarks described in \cite{stern2019}, in addition to being analysed in terms of complexity and compared to both an enumerative approach that solves all matchings using regular ICTS and the algorithm described in \cite{ma2016}. Lastly, completeness of the algorithm is considered as this is an important property in the context of MAPFM as a TUSS relaxation. Regarding the relevance of this work, it should be noted that besides TUSS, there are other applications such as planning warehouse robots \cite{wurman2007} that also could benefit from novel algorithms for MAPFM.
\printbibliography

\end{document}
